\documentclass[a4paper, 12pt]{article}
\usepackage{template} % Load your custom thesis package
\usepackage[backend=biber,style=numeric,sorting=none, url=false,maxnames=5,minnames=1]{biblatex}
\addbibresource{references.bib} % Your .bib file
\defbibheading{bibliography}{} % Remove the heading of bibliography

\begin{document}

% Title Page
\mytitlepage{Breve Introducción a Python}{Programación II}{Ciencia de los Datos}{Jean Charly Joseph Saint}{Valerie De la cruz Medina}{2023-1524}{2024 - C3} % Define title page based on your information

% Table of Contents
\clearpage
\tableofcontents
\listoffigures
\listoftables
\clearpage

% Reapply the page style to ensure consistency
\pagestyle{fancy}
\pagenumbering{arabic} % Switch to Arabic numerals for the main content

% Introduction
\section{Introducción}
Python es un lenguaje de programación potente y fácil de aprender. Tiene estructuras de datos de alto nivel eficientes y un simple pero efectivo sistema de programación orientado a objetos. La elegante sintaxis de Python y su tipado dinámico, junto a su naturaleza interpretada lo convierten en un lenguaje ideal para scripting y desarrollo rápido de aplicaciones en muchas áreas, para la mayoría de plataformas.\cite{DocumentationPython}
\\
\\
En este documento veremos los conceptos básicos de Python, como los tipos de datos y variables, las estructuras de control y una reflexión personal sobre el uso de Python en la ciencia de los datos.

use \textbackslash section for HEADING ONE (1.XXX)\\
use \textbackslash subsection for HEADING TWO (1.1.XXX)\\
use \textbackslash subsubsection for HEADING THREE (1.1.1.XXX)

% Uso Basico de Python
\clearpage
\section{Uso Básico de Python}
THIS IS THE SECTION FOR RESULTS

% Tipos de Datos y Variables
\clearpage
\section{Tipos de Datos y Variables}
THIS IS THE SECTION FOR THE CONCLUSION

% Estructuras de Control
\clearpage
\section{Estructuras de Control}
THIS IS THE SECTION FOR MATERIALS AND METHODS

% Reflexión Personal
\clearpage
\section{Reflexión Personal}
THIS IS THE SECTION FOR MATERIALS AND METHODS

% Bibliografia
\clearpage
\section{Bibliografia}
\printbibliography

\end{document}