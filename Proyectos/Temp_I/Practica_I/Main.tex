\documentclass[a4paper, 12pt]{article}
\usepackage{template} % Load your custom thesis package
\usepackage{minted} % Code highlighting
\usepackage[backend=biber,style=numeric,sorting=none, url=false,maxnames=5,minnames=1]{biblatex}
\addbibresource{references.bib} % Your .bib file
\defbibheading{bibliography}{} % Remove the heading of bibliography

\begin{document}

% Title Page
\mytitlepage{Breve Introducción a Python}{Programación II}{Ciencia de los Datos}{Jean Charly Joseph Saint}{Valerie De la cruz Medina}{2023-1524}{2024 - C3} % Define title page based on your information

% Table of Contents
\clearpage
\tableofcontents
\listoffigures
\listoftables
\clearpage

% Reapply the page style to ensure consistency
\pagestyle{fancy}
\pagenumbering{arabic} % Switch to Arabic numerals for the main content

% Introduction
\section{Introducción}
Python es un lenguaje de programación potente y fácil de aprender. Tiene estructuras de datos de alto nivel eficientes y un simple pero efectivo sistema de programación orientado a objetos. La elegante sintaxis de Python y su tipado dinámico, junto a su naturaleza interpretada lo convierten en un lenguaje ideal para scripting y desarrollo rápido de aplicaciones en muchas áreas, para la mayoría de plataformas.\cite{DocumentationPython}
\\
\\
En este documento veremos los conceptos básicos de Python, como los tipos de datos y variables, las estructuras de control y una reflexión personal sobre el uso de Python en la ciencia de los datos.

use \textbackslash section for HEADING ONE (1.XXX)\\
use \textbackslash subsection for HEADING TWO (1.1.XXX)\\
use \textbackslash subsubsection for HEADING THREE (1.1.1.XXX)

% Uso Basico de Python
\clearpage
\section{Uso Básico de Python}
En este capítulo se abordan los conceptos basicos para comenzar a programar en Python. Esto incluye la instalación del lenguaje, el uso de la consola interactiva y la ejecución de scripts. Ademas, se presentan ejemplos basicos de sintaxis como la impresión de texto, la declaración de variables y cómo utilizar comentarios en el código.

\section*{2.1 Instalación de Python}
Para instalar Python en su computadora, visite el sitio web oficial de Python en \url{https://www.python.org/}. Desde allí, puede descargar el instalador para su sistema operativo y seguir las instrucciones de instalación.
\\
\\
\textbf{Consola Interactiva}\\
Python viene con una consola interactiva que le permite escribir y ejecutar código de Python en tiempo real. Para abrir la consola interactiva, simplemente escriba \texttt{python} en su terminal o símbolo del sistema.
\\
\\
\textbf{Consola Interactiva}\\
Python viene con una consola interactiva que le permite escribir y ejecutar código de Python en tiempo real. Para abrir la consola interactiva, simplemente escriba \texttt{python} en su terminal o símbolo del sistema.
\\
\\
\textbf{Ejemplos de Sintaxis}\\
A continuación se presentan algunos ejemplos de sintaxis básica en Python:

\begin{minted}[frame=lines, bgcolor=lightgray, linenos]{python}
# Imprimir texto en la consola
print("Hola, mundo!")

# Declarar una variable
x = 5
\end{minted}

% Tipos de Datos y Variables
\clearpage
\section{Tipos de Datos y Variables}
THIS IS THE SECTION FOR THE CONCLUSION

% Estructuras de Control
\clearpage
\section{Estructuras de Control}
THIS IS THE SECTION FOR MATERIALS AND METHODS

% Reflexión Personal
\clearpage
\section{Reflexión Personal}
THIS IS THE SECTION FOR MATERIALS AND METHODS

% Bibliografia
\clearpage
\section{Bibliografia}
\printbibliography

\end{document}